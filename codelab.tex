\documentclass{beamer}

\usepackage[utf8x]{inputenc}
\usepackage{amssymb}
\usepackage{amsmath}
\usepackage{xcolor}
\usepackage{graphicx}
\usepackage{booktabs}
\usepackage{hyperref}
\usepackage[francais]{babel}

\graphicspath{{./images/}}

\usetheme{metropolis}
\title{The Deep Learning Codelab}
\date{19 octobre 2017}
\author{Jeff Abrahamson \and Hugo Mougard}
\institute{DevFest Nantes 2017}

% Let me use colours by name.
\newcommand\blue[1]{\textcolor{blue}{#1}}
\newcommand\red[1]{\textcolor{red}{#1}}
\newcommand\green[1]{\textcolor{green}{#1}}
\newcommand\gray[1]{\textcolor{gray}{#1}}
\newcommand\purple[1]{\textcolor{purple}{#1}}
\newcommand\smallgray[1]{\textcolor{gray}{\footnotesize\it #1}}
\newcommand\prevwork[1]{\smallgray{#1}}

% Some ways to include images.
% o=only, f=full, h=fit height.
% more g's or h's means smaller still.
\newcommand\cimgo[1]{\vfill\centerline{\includegraphics{#1}}\vfill}
\newcommand\cimgff[1]{\vfill\centerline{\includegraphics[width=1.2\textwidth]{#1}}\vfill}
\newcommand\cimgf[1]{\vfill\centerline{\includegraphics[width=\textwidth]{#1}}\vfill}
\newcommand\cimg[1]{\vfill\centerline{\includegraphics[width=.9\textwidth]{#1}}\vfill}
\newcommand\cimgg[1]{\vfill\centerline{\includegraphics[width=.8\textwidth]{#1}}\vfill}
\newcommand\cimggg[1]{\vfill\centerline{\includegraphics[width=.7\textwidth]{#1}}\vfill}
\newcommand\cimgsm[1]{\vfill\centerline{\includegraphics[width=.4\textwidth]{#1}}\vfill}
\newcommand\cimgt[1]{\centerline{\includegraphics[width=.2\textwidth]{#1}}\vfill}
\newcommand\cimgh[1]{\vfill\centerline{\includegraphics[height=.9\textheight]{#1}}\vfill}
\newcommand\cimghh[1]{\vfill\centerline{\includegraphics[height=.8\textheight]{#1}}\vfill}
\newcommand\cimghhh[1]{\vfill\centerline{\includegraphics[height=.7\textheight]{#1}}\vfill}
\newcommand\cimghhhh[1]{\vfill\centerline{\includegraphics[height=.6\textheight]{#1}}\vfill}

% Just one phrase centered horizontally and vertically.
\newcommand\vphrase[1]{\vfill\centerline{\large\bf\blue{#1}}\vfill}
% Just one phrase, I'll position it.
\newcommand\phrase[1]{\centerline{\huge\bf\blue{#1}}}

\begin{document}
\maketitle

\section{github}

\begin{frame}
\vfill
\centerline{\large\bf\blue{\url{https://github.com/m09/deeplearning-codelab}}}
\vspace{5mm}
\centerline{\large\bf\blue{\url{https://github.com/m09/deeplearning-codelab-image}}}
\vfill
\end{frame}

\begin{frame}{For the tutorials}
  \begin{enumerate}
  \item Make a Github account (if you don't already have one)
  \item Go to \url{http://35.189.251.190}
  \end{enumerate}
\end{frame}

\section{Machine Learning}
\label{sec:ml}
\begin{frame}{Definition}
  \begin{itemize}
  \item Supervised
  \item Unsupervised
  \item Reinforcement
  \end{itemize}
\end{frame}

\begin{frame}{What is ML?}
  \only<1>{
    \vphrase{Machine learning is not magic}
  }
  \only<2>{
    \vphrase{Machine learning is mathematics}
  }
  \only<3>{
    \vspace{1cm}
    \blue{\bf Mostly, it's these maths:}
    \begin{itemize}
    \item Probability
    \item Statistics
    \item Linear algebra
    \item Optimisation theory
    \item Differential calculus
    \end{itemize}
  }
\end{frame}

\begin{frame}[t]
  \frametitle{What is Statistics}
  
  \begin{enumerate}
  \item<1-3> Identify a question or problem.
  \item<1-3> Collect relevant data on the topic.
  \item<1-3> Analyze the data.
  \item<1-3> Form a conclusion.
  \end{enumerate}
  \only<2>{Sadly, sometimes people forget 1.}
  \only<3>{Statistics is about making 2--4 efficient, rigorous, and meaningful.}
  \only<3>{\vfill\prevwork{\textit{OpenIntro Statistics},
      2nd edition, D.~Diez, C.~Barr, M.~Çetinkaya-Rundel, 2013.}}
\end{frame}

\begin{frame}[t]
  \frametitle{What is data science?}

  \only<1-4>{(Exercise: Is this the same question as the last slide?)}

  \only<1>{
    \begin{enumerate}
    \item Define the question of interest
    \item Get the data
    \item Clean the data
    \item Explore the data
    \item Fit statistical models
    \item Communicate the results
    \item Make your analysis reproducible
    \end{enumerate}
  }
  \only<2>{
    \begin{enumerate}
    \item Define the question of interest
    \item Get the data
    \item Clean the data
    \item \red{Explore the data}
    \item \red{Fit statistical models}
    \item Communicate the results
    \item Make your analysis reproducible
    \end{enumerate}

    \blue{What the public thinks.}
  }
  \only<3>{
    \begin{enumerate}
    \item Define the question of interest
    \item \red{Get the data}
    \item \red{Clean the data}
    \item Explore the data
    \item Fit statistical models
    \item \red{Communicate the results}
    \item \red{Make your analysis reproducible}
    \end{enumerate}

    \blue{Where we spend most of our time.}
  }
  \only<4>{
    \begin{enumerate}
    \item \red{Define the question of interest}
    \item Get the data
    \item Clean the data
    \item Explore the data
    \item Fit statistical models
    \item Communicate the results
    \item Make your analysis reproducible
    \end{enumerate}

    \blue{The easiest part to forget.}
  }
  \only<5>{\vfill\prevwork{\url{http://simplystatistics.org/2015/03/17/} \url{data-science-done-well-looks-easy-and-that-is-a-big-} \url{problem-for-data-scientists/}}}

  \only<6>{\vfill\cimg{images/model-in-one-day.jpg}\vfill}
\end{frame}

\begin{frame}{Representation}
  \vphrase{Typically a vector space}

  \vphrase{Features are dimensions}
\end{frame}

\begin{frame}{Features}
  \vphrase{Feature extraction}

  \vphrase{Feature engineering, synthetic features}
\end{frame}

\begin{frame}{Feature Engineering}
  \begin{enumerate}
  \item Brainstorm
  \item Pick some
  \item Make them
  \item Evaluate
  \item Repeat
  \end{enumerate}
\end{frame}

\section{Artificial Neural Networks}
\label{sec:nn}

\begin{frame}{Neural Networks}
  \textbf{Inspired} by brain :
  \begin{itemize}
  \item 1 neuron = 1 simple computation
  \item 10M neurons = 10M simple computations
  \item 10M combined neurons = 10M simple computations combined = 1
    interesting computation
  \end{itemize}
\end{frame}

\begin{frame}{Neural Networks --- Neural Network Example}
  \cimgg{neural-network.png}
\end{frame}

\begin{frame}{Neural Networks --- Biological Neuron}
  \cimgg{neuron.png}
\end{frame}

\begin{frame}{Neural Networks --- Artificial Neuron}
  \cimgg{neuron-model.png}
\end{frame}

\begin{frame}{Neural Networks --- Why do we need activations?}
  \cimgg{xor.png}
\end{frame}

\begin{frame}{Neural Networks --- Activation Function: ReLU}
  \cimgg{relu.png}
\end{frame}

\begin{frame}{Neural Networks --- Activation Function: Sigmoid}
  \cimgg{activation.jpeg}
\end{frame}

\begin{frame}{Neural Networks --- So What's a Neuron Again?}
  \begin{itemize}
  \item   Neuron = Weighted averaged of its input + activation function
  \item   Goal of learning = learn good weights.
  \end{itemize}
\end{frame}

\begin{frame}{Neural Networks --- Neural Network Example Comeback}
  \cimgg{neural-network.png}
\end{frame}

\begin{frame}{Neural Networks --- How to know if they're good}
  \begin{itemize}
  \item Find some good examples
  \item Compare network output and good examples (Loss function).
  \end{itemize}
\end{frame}

\begin{frame}{Neural Networks --- How to Learn Weights?}
  \begin{enumerate}
  \item A Neural Network loss is a function
  \item You actually know how to minimize functions (derivatives)
  \item …
  \item Profit
  \end{enumerate}
\end{frame}

\section{Convolutionnal Neural Networks}
\label{sec:convnets}
\begin{frame}{CNNs --- Aren't Simple Nets Enough?}
  \vphrase{Not for images. Each neuron receives each pixel.}
  \vphrase{→ Too Much Information}
  \vfill
  \cimgg{tmi.jpg}
  \vfill
\end{frame}
\begin{frame}{CNNs --- TMI 1/3}
  \cimgg{park.jpg}
\end{frame}
\begin{frame}{CNNs --- TMI 2/3}
  \cimgg{station.jpg}
\end{frame}
\begin{frame}{CNNs --- TMI 3/3}
  \cimgg{street.jpg}
\end{frame}
\begin{frame}{CNNs --- Solution 1/3}
  \vphrase{Restrict neurons' receptive fields}
  \cimggg{kernel.jpg}
\end{frame}
\begin{frame}{CNNs --- Solution 2/3}
  \vphrase{Pooling: propagate salient info}
  \cimg{pooling.png}
\end{frame}
\begin{frame}{CNNs --- Solution 3/3}
  \vphrase{Combine them hierarchically}
  \cimg{cnn.jpg}
\end{frame}
\begin{frame}{CNNs --- Examples}
  \cimg{hierarchical-features.png}
\end{frame}


\section{Infrastructure}
\label{sec:infra}
\begin{frame}{Infrastructure --- Google Cloud}
  \cimgsm{google-cloud.png}
\end{frame}
\begin{frame}{Infrastructure --- Kubernetes}
  \cimgsm{kubernetes.png}
\end{frame}
\begin{frame}{Infrastructure --- Docker}
  \cimgsm{docker.jpg}
\end{frame}
\begin{frame}{Infrastructure --- Jupyter}
  \cimgsm{jupyter.png}
\end{frame}
\begin{frame}{Infrastructure --- Python}
  \cimgsm{python.png}
\end{frame}
\begin{frame}{Infrastructure --- Tensorflow}
  \cimgsm{tensorflow.png}
\end{frame}
\begin{frame}{Infrastructure --- Keras}
  \cimgsm{keras.jpg}
\end{frame}
\begin{frame}{Infrastructure --- Scikit Learn}
  \cimgsm{scikit.png}
\end{frame}
\begin{frame}{Infrastructure --- In Practice}
  \begin{enumerate}
  \item Make a Github account (if you don't already have one)
  \item Go to \url{http://35.189.251.190}
  \end{enumerate}
\end{frame}


\section{Keras}
\label{sec:keras}
\begin{frame}{Start with a plan}
\begin{enumerate}
\item Load Data.
\item Define Model.
\item Compile Model.
\item Fit Model.
\item Evaluate Model.
\item Tie It All Together.
\end{enumerate}
\end{frame}

\begin{frame}{Code!}
  \vphrase{Pima Indians and Diabetes}
\end{frame}

\begin{frame}{Code!}
  \vphrase{MNIST}
\end{frame}

\begin{frame}{Code!}
  \vphrase{Imagenet}
\end{frame}

% \section{MNIST}
% \label{sec:mnist}
% \begin{frame}{Hey I'm a frame}
  
% \end{frame}

% \section{Model}
% \label{sec:model}
% \begin{frame}{Hey I'm a frame}
%   Hello, world!
% \end{frame}

\section{Resources}
\label{sec:resources}

\begin{frame}{ML Week}
  \frametitle{Resources}
  \cimggg{ml-week.png}

  \vspace{5mm}
  \centerline{\url{http://www.ml-week.com/}}
\end{frame}

\begin{frame}
  \frametitle{The Basics}

  \begin{itemize}
  \item Everything is a math problem (and a vector space)
  \item Features are king (and most of our time)
  \item More data is usually better
  \end{itemize}
\end{frame}

\begin{frame}
  \frametitle{The Choices (I)}
    
  \begin{itemize}
  \item Supervised (I know some answers)
  \item Unsupervised (I don't know much)
  \end{itemize}
\end{frame}

\begin{frame}
  \frametitle{The Choices (II)}
  \begin{itemize}
  \item Regression (answers in $\mathbb{R}$)
  \item Classification (answers are categories)
  \end{itemize}
\end{frame}

\begin{frame}
  \frametitle{The Tools}

  \phrase{There are many.}

  \vspace{1cm}
  \centerline{  \red{It's easy to waste time}}
\end{frame}

\begin{frame}
  \frametitle{Techniques (I)}

  \begin{itemize}
  \item Linear (and non-linear) regression
  \item Logistic regression
  \item SVM
  \end{itemize}

  Under the hood, it's almost always a \purple{cost function} and
  \purple{gradient descent}.
\end{frame}

\begin{frame}
  \frametitle{Techniques (II): Neural networks}

  \begin{itemize}
  \item Feed forward
  \item Backpropagation
  \item Deep
  \item Feature learning
  \end{itemize}
\end{frame}

\begin{frame}
  \frametitle{Techniques (III)}

  \begin{itemize}
  \item KMeans
  \item PCA
  \item Recommendation
  \end{itemize}
\end{frame}

\begin{frame}
  \frametitle{The Future}

  \begin{itemize}
  \item Learning
  \item Algorithms about learning
  \item Algorithms about features!
  \end{itemize}
\end{frame}

\begin{frame}
  \frametitle{Resources}

  \begin{itemize}
  \item Nantes Machine Learning Meetup
  \item Rennes Machine Learning Meetup \textit{\blue{(nouveau !)}}
  \item Community
  \item Conversation
  \end{itemize}
\end{frame}

\begin{frame}
  \frametitle{Resources}
  \cimggg{rmlm.png}

  \vspace{5mm}
  \centerline{\url{https://www.meetup.com/Meetup-Machine-Learning-Rennes/}}
\end{frame}

\begin{frame}
  \frametitle{Resources}
  \cimggg{nmlm.png}

  \vspace{5mm}
  \centerline{\url{http://www.meetup.com/Nantes-Machine-Learning-Meetup/}}
\end{frame}

\begin{frame}
  \frametitle{Resources}
  \cimggg{ml-week.png}

  \vspace{5mm}
  \centerline{\url{http://www.ml-week.com/maintenant.html}}
  \vspace{2mm}
  \centerline{\purple{Remise : DEVFEST}}
\end{frame}

\section{Q \& A}
\label{sec:qa}

\begin{frame}
  % https://www.pexels.com/photo/food-healthy-people-woman-41219/
  % https://static.pexels.com/photos/41219/apple-diet-face-food-41219.jpeg
  % CC0 license
  \cimgh{apple-pear.jpg}
  \vspace{-.25\textheight}
  \phrase{Questions?}
  \vspace{.25\textheight}
\end{frame}

\end{document}
%%% Local Variables:
%%% mode: latex
%%% TeX-master: t
%%% End:
